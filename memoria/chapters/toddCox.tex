
En primer lugar, importamos las librerías que se utilizarán:
\begin{itemize}
    \item  \texttt{Group}: fichero principal de la librería. En él se encuentran las principales clases y funciones para definir los diferentes grupos y sirve para identificar y dar estructura de grupo al conjunto de generadores y relatores dados como entrada. 
    \item  \texttt{ToddCoxeter}: contiene la implementación del \textit{Algoritmo de Todd Coxeter} junto a todas las funcionalidades descritas anteriormente. 


    \begin{tcolorbox}[breakable, size=fbox, boxrule=1pt, pad at break*=1mm,colback=cellbackground, colframe=cellborder]
%\prompt{In}{incolor}{1}{\boxspacing}
\begin{Verbatim}[commandchars=\\\{\}]
\PY{k+kn}{from} \PY{n+nn}{Group} \PY{k+kn}{import} \PY{o}{*}
\PY{k+kn}{from} \PY{n+nn}{ToddCoxeter} \PY{k+kn}{import} \PY{n}{CosetTable}\PY{p}{,} \PY{n}{readGroup}
\end{Verbatim}
\end{tcolorbox}

\end{itemize}



El procedimiento a seguir se desarrollará a continuación y se ilustrará con el siguiente ejemplo sencillo:
\[
G=\langle a,b \mid a^2, b^2, ab=ba \rangle \quad y \quad H=\{1\}.
\]

\begin{enumerate}
\item Leemos los datos de entrada, ya sea mediante variables para definir el grupo o
  haciendo uso del método \textit{readGroup}, en el que se le ha de especificar la ruta del fichero donde está el grupo.
  
  \begin{tcolorbox}[breakable, size=fbox, boxrule=1pt, pad at break*=1mm,colback=cellbackground, colframe=cellborder]
\prompt{In}{incolor}{1}{\boxspacing}
\begin{Verbatim}[commandchars=\\\{\}]
\PY{n}{gen} \PY{o}{=} \PY{p}{[}\PY{l+s+s1}{\PYZsq{}}\PY{l+s+s1}{a}\PY{l+s+s1}{\PYZsq{}}\PY{p}{,}\PY{l+s+s1}{\PYZsq{}}\PY{l+s+s1}{b}\PY{l+s+s1}{\PYZsq{}}\PY{p}{]}
\PY{n}{rels} \PY{o}{=} \PY{p}{[}\PY{l+s+s1}{\PYZsq{}}\PY{l+s+s1}{aa}\PY{l+s+s1}{\PYZsq{}}\PY{p}{,}\PY{l+s+s1}{\PYZsq{}}\PY{l+s+s1}{bb}\PY{l+s+s1}{\PYZsq{}}\PY{p}{,}\PY{l+s+s1}{\PYZsq{}}\PY{l+s+s1}{abAB}\PY{l+s+s1}{\PYZsq{}}\PY{p}{]}
\PY{n}{genH} \PY{o}{=} \PY{p}{[}\PY{p}{]}
\end{Verbatim}
\end{tcolorbox}
  
  
  
\item Creamos un objeto de la clase \textit{CosetTable}. Después, llamamos al método \textit{CosetEnumeration} para aplicar el \textit{Algoritmo de Todd Coxeter}.

\begin{tcolorbox}[breakable, size=fbox, boxrule=1pt, pad at break*=1mm,colback=cellbackground, colframe=cellborder]
\prompt{In}{incolor}{2}{\boxspacing}
\begin{Verbatim}[commandchars=\\\{\}]
\PY{n}{G} \PY{o}{=} \PY{n}{CosetTable}\PY{p}{(}\PY{n}{gen}\PY{p}{,}\PY{n}{rels}\PY{p}{,} \PY{n}{genH}\PY{p}{)}
\PY{n}{G}\PY{o}{.}\PY{n}{CosetEnumeration}\PY{p}{(}\PY{p}{)}
\end{Verbatim}
\end{tcolorbox}


  
Podemos mostrar la tabla de clases laterales de $G/H$, que se obtiene mediante el método \textit{coset\_table}.

    \begin{tcolorbox}[breakable, size=fbox, boxrule=1pt, pad at break*=1mm,colback=cellbackground, colframe=cellborder]
\prompt{In}{incolor}{3}{\boxspacing}
\begin{Verbatim}[commandchars=\\\{\}]
\PY{n}{T} \PY{o}{=} \PY{n}{G}\PY{o}{.}\PY{n}{coset\PYZus{}table}\PY{p}{(}\PY{p}{)}
\PY{n+nb}{print}\PY{p}{(}\PY{n}{T}\PY{p}{)}
\end{Verbatim}
\end{tcolorbox}

    \begin{center}
    \adjustimage{max size={0.29\linewidth}{0.29\paperheight}}{img/1g.png}
    \end{center}


Como hemos comentado anteriormente, el número de filas sin contar las cabeceras indica el número de elementos del grupo $G/H$, que coincide con el índice $[G:H]$. En este ejemplo, se tiene:
\[
    [G:H] = \frac{|G|}{|H|} = \frac{|G|}{1} = |G| = 4 \: .
\]

\newpage
Equivalentemente, el grafo de Schreier asociado se llama de la siguiente forma:

    \begin{tcolorbox}[breakable, size=fbox, boxrule=1pt, pad at break*=1mm,colback=cellbackground, colframe=cellborder]
\prompt{In}{incolor}{4}{\boxspacing}
\begin{Verbatim}[commandchars=\\\{\}]
\PY{n}{G}\PY{o}{.}\PY{n}{schreier\PYZus{}graph}\PY{p}{(}\PY{n}{notes}\PY{o}{=}\PY{k+kc}{False}\PY{p}{)}
\end{Verbatim}
\end{tcolorbox}

    \begin{center}
    \adjustimage{max size={0.4\linewidth}{0.26\paperheight}}{img/code_7_0.png}
    \end{center}



\item El siguiente paso es el de obtener los generadores de Schreier y definir el grupo como aquel generado por estos elementos.
  
    \begin{tcolorbox}[breakable, size=fbox, boxrule=1pt, pad at break*=1mm,colback=cellbackground, colframe=cellborder]
\prompt{In}{incolor}{5}{\boxspacing}
\begin{Verbatim}[commandchars=\\\{\}]
\PY{k}{def} \PY{n+nf}{print\PYZus{}gens}\PY{p}{(}\PY{n}{gens}\PY{p}{)}\PY{p}{:}
    \PY{k}{for} \PY{n}{i} \PY{o+ow}{in} \PY{n+nb}{range}\PY{p}{(}\PY{n+nb}{len}\PY{p}{(}\PY{n}{gens}\PY{p}{)}\PY{p}{)}\PY{p}{:}
        \PY{n+nb}{print}\PY{p}{(}\PY{l+s+sa}{f}\PY{l+s+s2}{\PYZdq{}}\PY{l+s+s2}{g}\PY{l+s+si}{\PYZob{}}\PY{n}{i}\PY{l+s+si}{\PYZcb{}}\PY{l+s+s2}{ = }\PY{l+s+si}{\PYZob{}}\PY{n}{gens}\PY{p}{[}\PY{n}{i}\PY{p}{]}\PY{l+s+si}{\PYZcb{}}\PY{l+s+s2}{\PYZdq{}}\PY{p}{)}
        
\PY{n}{generators} \PY{o}{=} \PY{n}{G}\PY{o}{.}\PY{n}{getGenerators}\PY{p}{(}\PY{p}{)}
\PY{n}{print\PYZus{}gens}\PY{p}{(}\PY{n}{generators}\PY{p}{)}
\end{Verbatim}
\end{tcolorbox}  

 \begin{tcolorbox}[breakable, size=fbox, boxrule=.5pt, pad at break*=1mm, opacityfill=0]
\prompt{Out}{outcolor}{5}{\boxspacing}
    \begin{Verbatim}[commandchars=\\\{\}]
g0 = (1, 2)(3, 4)
g1 = (1, 3)(2, 4)
    \end{Verbatim}
\end{tcolorbox}
    

En este momento, tenemos los generadores que definen al grupo. En particular, en este ejemplo el grupo estará definido por
$G = \langle \, g0, \, g1 \, \rangle $.


\begin{tcolorbox}[breakable, size=fbox, boxrule=1pt, pad at break*=1mm,colback=cellbackground, colframe=cellborder]
\prompt{In}{incolor}{6}{\boxspacing}
\begin{Verbatim}[commandchars=\\\{\}]
\PY{n}{Gr} \PY{o}{=} \PY{n}{Group(elems=generators)}
\end{Verbatim}
\end{tcolorbox}  
 
 
\item  Usar el método \textit{is\_isomorphic} para identificar cada grupo con
  grupos conocidos.
  
  
Observando la presentación dada, no es dificil ver que el grupo debe ser isomorfo al Grupo de Klein:

\begin{tcolorbox}[breakable, size=fbox, boxrule=1pt, pad at break*=1mm,colback=cellbackground, colframe=cellborder]
\prompt{In}{incolor}{7}{\boxspacing}
\begin{Verbatim}[commandchars=\\\{\}]
\PY{n}{K} \PY{o}{=} \PY{n}{KleinGroup}\PY{p}{(}\PY{p}{)}
\PY{n}{Gr}\PY{o}{.}\PY{n}{is\PYZus{}isomorphic}\PY{p}{(}\PY{n}{K}\PY{p}{)}
\end{Verbatim}
\end{tcolorbox}

\begin{tcolorbox}[breakable, size=fbox, boxrule=.5pt, pad at break*=1mm, opacityfill=0]
\prompt{Out}{outcolor}{7}{\boxspacing}
\begin{Verbatim}[commandchars=\\\{\}]
True
\end{Verbatim}
\end{tcolorbox}
  
  
\end{enumerate}

 
\newpage
\subsection{Ejemplos}

En la siguiente sección se estudiarán diferentes ejemplos de grupos finitamente presentados. El objetivo es identificar la presentación dada estableciendo isomorfismos con grupos estudiados.

Consideramos el siguiente grupo, que se encuentra en \textit{Groups/1.txt}.
\begin{align*} \label{1st}
    G = \langle a,b\; | \; ab^{-1}b^{-1}a^{-1}bbb, b^{-1}a^{-1}a^{-1}baaa\rangle \quad y \quad H = \{ 1\}.
\end{align*}

\begin{enumerate}
En primer lugar, usamos el método \textit{readGroup} para leerlo:


    \begin{tcolorbox}[breakable, size=fbox, boxrule=1pt, pad at break*=1mm,colback=cellbackground, colframe=cellborder]
\prompt{In}{incolor}{1}{\boxspacing}
\begin{Verbatim}[commandchars=\\\{\}]
\PY{n}{file} \PY{o}{=} \PY{l+s+s2}{\PYZdq{}}\PY{l+s+s2}{Groups/1.txt}\PY{l+s+s2}{\PYZdq{}}
\PY{n}{f} \PY{o}{=} \PY{n}{readGroup}\PY{p}{(}\PY{n}{file}\PY{p}{)}
\PY{n+nb}{print}\PY{p}{(}\PY{n}{f}\PY{p}{)}
\end{Verbatim}
\end{tcolorbox}

    \begin{Verbatim}[commandchars=\\\{\}]
(['a', 'b'], ['aBBAbbb', 'BAAbaaa'], [])
    \end{Verbatim}

De igual modo que en el ejemplo anterior, aplicamos el \textit{Algoritmo de Todd Coxeter}. 

\begin{tcolorbox}[breakable, size=fbox, boxrule=1pt, pad at break*=1mm,colback=cellbackground, colframe=cellborder]
\prompt{In}{incolor}{2}{\boxspacing}
\begin{Verbatim}[commandchars=\\\{\}]
\PY{n}{G} \PY{o}{=} \PY{n}{CosetTable}\PY{p}{(}\PY{n}{file}\PY{p}{)}
\PY{n}{G}\PY{o}{.}\PY{n}{CosetEnumeration}\PY{p}{(}\PY{p}{)}
\end{Verbatim}
\end{tcolorbox}




A continuación, mostramos la tabla de clases de $G/H$:

    \begin{tcolorbox}[breakable, size=fbox, boxrule=1pt, pad at break*=1mm,colback=cellbackground, colframe=cellborder]
\prompt{In}{incolor}{3}{\boxspacing}
\begin{Verbatim}[commandchars=\\\{\}]
\PY{n+nb}{print}\PY{p}{(}\PY{n}{G}\PY{o}{.}\PY{n}{table}\PY{p}{)}
\end{Verbatim}
\end{tcolorbox}

    \begin{center}
    \adjustimage{max size={0.28\linewidth}{0.28\paperheight}}{img/2g.png}
    \end{center}

Como vemos, únicamente hay una fila de clases, por lo que $[G:H]=1$ y, como $H=\{1\}$, $G$ debe ser necesariamente el grupo trivial. Como consecuencia, el grafo de Schreier posee un único vértice.
En este ejemplo, no es de utilidad darle estructura de grupo a $G$, en cambio, servirá para mostrar un ejemplo de un grupo sencillo en el que el número de clases que se usan es muy elevado. 

    \begin{tcolorbox}[breakable, size=fbox, boxrule=1pt, pad at break*=1mm,colback=cellbackground, colframe=cellborder]
\prompt{In}{incolor}{4}{\boxspacing}
\begin{Verbatim}[commandchars=\\\{\}]
\PY{n}{u} \PY{o}{=} \PY{n}{G}\PY{o}{.}\PY{n}{usedCosets}\PY{p}{(}\PY{p}{)}
\PY{n}{f} \PY{o}{=} \PY{n}{G}\PY{o}{.}\PY{n}{finalCosets}\PY{p}{(}\PY{p}{)}

\PY{n+nb}{print}\PY{p}{(}\PY{l+s+s2}{\PYZdq{}}\PY{l+s+s2}{Clases usadas: }\PY{l+s+si}{\PYZob{}\PYZcb{}}\PY{l+s+s2}{ }\PY{l+s+se}{\PYZbs{}n}\PY{l+s+s2}{ Clases vivas: }\PY{l+s+si}{\PYZob{}\PYZcb{}}\PY{l+s+s2}{\PYZdq{}}\PY{o}{.}\PY{n}{format}\PY{p}{(}\PY{n}{u}\PY{p}{,}\PY{n}{f}\PY{p}{)}\PY{p}{)}
\end{Verbatim}
\end{tcolorbox}


\begin{tcolorbox}[breakable, size=fbox, boxrule=.5pt, pad at break*=1mm, opacityfill=0]
\prompt{Out}{outcolor}{8}{\boxspacing}
    \begin{Verbatim}[commandchars=\\\{\}]
Clases usadas: 85
Clases vivas: 1
\end{Verbatim}
\end{tcolorbox}


\newpage
El siguiente grupo $G$ se encuentra definido en \textit{Groups/S3\_2.txt}, y como subgrupo $H\leq G$, tomamos aquel generado por $a$.
\begin{align*}
    G = \langle a,b \; | \; a^2 = b^2 = 1, (ab)^3=1\rangle \quad y \quad H= \langle a \rangle.
\end{align*}



    \begin{tcolorbox}[breakable, size=fbox, boxrule=1pt, pad at break*=1mm,colback=cellbackground, colframe=cellborder]
\prompt{In}{incolor}{1}{\boxspacing}
\begin{Verbatim}[commandchars=\\\{\}]
\PY{n}{file} \PY{o}{=} \PY{l+s+s2}{\PYZdq{}}\PY{l+s+s2}{Groups/S3\PYZus{}2.txt}\PY{l+s+s2}{\PYZdq{}}
\PY{n}{f} \PY{o}{=} \PY{n}{readGroup}\PY{p}{(}\PY{n}{file}\PY{p}{)}
\PY{n}{f}
\end{Verbatim}
\end{tcolorbox}

            \begin{tcolorbox}[breakable, size=fbox, boxrule=.5pt, pad at break*=1mm, opacityfill=0]
\prompt{Out}{outcolor}{2}{\boxspacing}
\begin{Verbatim}[commandchars=\\\{\}]
(['a', 'b'], ['aa', 'bb', 'ababab'], ['a'])
\end{Verbatim}
\end{tcolorbox}
        
Mostramos la tabla de clases de $G/H$ resultante tras aplicar el algoritmo.
    \begin{tcolorbox}[breakable, size=fbox, boxrule=1pt, pad at break*=1mm,colback=cellbackground, colframe=cellborder]
\prompt{In}{incolor}{3}{\boxspacing}
\begin{Verbatim}[commandchars=\\\{\}]
\PY{n+nb}{print}\PY{p}{(}\PY{n}{G}\PY{o}{.}\PY{n}{coset\PYZus{}table}\PY{p}{(}\PY{p}{)}\PY{p}{)}
\end{Verbatim}
\end{tcolorbox}

    \begin{center}
    \adjustimage{max size={0.28\linewidth}{0.28\paperheight}}{img/4g.png}
    \end{center}



Ahora, el grafo de Schreier equivalente:
    \begin{tcolorbox}[breakable, size=fbox, boxrule=1pt, pad at break*=1mm,colback=cellbackground, colframe=cellborder]
\prompt{In}{incolor}{4}{\boxspacing}
\begin{Verbatim}[commandchars=\\\{\}]
\PY{n}{G}\PY{o}{.}\PY{n}{schreier\PYZus{}graph}\PY{p}{(}\PY{n}{notes}\PY{o}{=}\PY{k+kc}{False}\PY{p}{)}
\end{Verbatim}
\end{tcolorbox}


    \begin{center}
    \adjustimage{max size={0.2\linewidth}{0.2\paperheight}}{img/code_48_0.png}
    \end{center}

Obtenemos los generadores de Schreier que definen al grupo, que se usarán, por el Teorema \ref{important}, para darle estructura de grupo de Permutaciones.

    \begin{tcolorbox}[breakable, size=fbox, boxrule=1pt, pad at break*=1mm,colback=cellbackground, colframe=cellborder]
\prompt{In}{incolor}{5}{\boxspacing}
\begin{Verbatim}[commandchars=\\\{\}]
\PY{n}{generators} \PY{o}{=} \PY{n}{G}\PY{o}{.}\PY{n}{getGenerators}\PY{p}{(}\PY{p}{)}
\PY{n}{print\PYZus{}gens}\PY{p}{(}\PY{n}{generators}\PY{p}{)}

\PY{n}{S} \PY{o}{=} \PY{n}{Group}\PY{p}{(}\PY{n}{elems}\PY{o}{=}\PY{n}{generators}\PY{p}{)}
\end{Verbatim}
\end{tcolorbox}


\begin{tcolorbox}[breakable, size=fbox, boxrule=.5pt, pad at break*=1mm, opacityfill=0]
\prompt{Out}{outcolor}{5}{\boxspacing}
    \begin{Verbatim}[commandchars=\\\{\}]
g0 = (2, 3)
g1 = (1, 2)
    \end{Verbatim}
\end{tcolorbox}

Vemos que el grupo es isomorfo a $S_3$, el grupo Simétrico de orden $6$.

    \begin{tcolorbox}[breakable, size=fbox, boxrule=1pt, pad at break*=1mm,colback=cellbackground, colframe=cellborder]
\prompt{In}{incolor}{6}{\boxspacing}
\begin{Verbatim}[commandchars=\\\{\}]
\PY{n}{S3} \PY{o}{=} \PY{n}{SymmetricGroup}\PY{p}{(}\PY{l+m+mi}{3}\PY{p}{)}
\PY{n}{S3}\PY{o}{.}\PY{n}{is\PYZus{}isomorphic}\PY{p}{(}\PY{n}{S}\PY{p}{)}
\end{Verbatim}
\end{tcolorbox}

            \begin{tcolorbox}[breakable, size=fbox, boxrule=.5pt, pad at break*=1mm, opacityfill=0]
\prompt{Out}{outcolor}{6}{\boxspacing}
\begin{Verbatim}[commandchars=\\\{\}]
True
\end{Verbatim}
\end{tcolorbox}
        

Otra forma equivalente para construir el grupo $S_3$ es mediante el producto
semidirecto $A_3 \rtimes K$, donde $K = \{1,(12)\}$. Véase el Ejemplo \ref{SnAn}.

    \begin{tcolorbox}[breakable, size=fbox, boxrule=1pt, pad at break*=1mm,colback=cellbackground, colframe=cellborder]
\prompt{In}{incolor}{7}{\boxspacing}
\begin{Verbatim}[commandchars=\\\{\}]
\PY{n}{A} \PY{o}{=} \PY{n}{AlternatingGroup}\PY{p}{(}\PY{l+m+mi}{3}\PY{p}{)}
\PY{n}{B} \PY{o}{=} \PY{n}{SymmetricGroup}\PY{p}{(}\PY{l+m+mi}{2}\PY{p}{)}

\PY{n+nb}{print}\PY{p}{(}\PY{n}{A}\PY{p}{,} \PY{n}{B}\PY{p}{)}
\end{Verbatim}
\end{tcolorbox}


\begin{tcolorbox}[breakable, size=fbox, boxrule=.5pt, pad at break*=1mm, opacityfill=0]
\prompt{Out}{outcolor}{7}{\boxspacing}
    \begin{Verbatim}[commandchars=\\\{\}]
Group with 3 elements: \{(1, 2, 3), (), (1, 3, 2)\} 
Group with 2 elements: \{(), (1, 2)\}
    \end{Verbatim}
\end{tcolorbox}

Para construir este producto semidirecto, habrá que ver las posibles acciones:
\[
    \varphi \colon K \to \operatorname{Aut(A_3)}
\]


    \begin{tcolorbox}[breakable, size=fbox, boxrule=1pt, pad at break*=1mm,colback=cellbackground, colframe=cellborder]
\prompt{In}{incolor}{8}{\boxspacing}
\begin{Verbatim}[commandchars=\\\{\}]
\PY{n}{AutA} \PY{o}{=} \PY{n}{A}\PY{o}{.}\PY{n}{AutomorphismGroup}\PY{p}{(}\PY{p}{)}
\PY{n}{Hom} \PY{o}{=} \PY{n}{B}\PY{o}{.}\PY{n}{AllHomomorphisms}\PY{p}{(}\PY{n}{AutA}\PY{p}{)}
\end{Verbatim}
\end{tcolorbox}

    \begin{tcolorbox}[breakable, size=fbox, boxrule=1pt, pad at break*=1mm,colback=cellbackground, colframe=cellborder]
\prompt{In}{incolor}{9}{\boxspacing}
\begin{Verbatim}[commandchars=\\\{\}]
\PY{n}{hom0} \PY{o}{=} \PY{n}{GroupHomomorphism}\PY{p}{(}\PY{n}{B}\PY{p}{,} \PY{n}{AutA}\PY{p}{,} \PY{k}{lambda} \PY{n}{x}\PY{p}{:}\PY{n}{Hom}\PY{p}{[}\PY{l+m+mi}{0}\PY{p}{]}\PY{p}{(}\PY{n}{x}\PY{p}{)}\PY{p}{)} 
\PY{n}{hom1} \PY{o}{=} \PY{n}{GroupHomomorphism}\PY{p}{(}\PY{n}{B}\PY{p}{,} \PY{n}{AutA}\PY{p}{,} \PY{k}{lambda} \PY{n}{x}\PY{p}{:}\PY{n}{Hom}\PY{p}{[}\PY{l+m+mi}{1}\PY{p}{]}\PY{p}{(}\PY{n}{x}\PY{p}{)}\PY{p}{)}

\PY{n}{SP0} \PY{o}{=} \PY{n}{A}\PY{o}{.}\PY{n}{semidirect\PYZus{}product}\PY{p}{(}\PY{n}{B}\PY{p}{,}\PY{n}{hom0}\PY{p}{)} \PY{l+m+mi}{#hom0 es la acción trivial}
\PY{n}{SP1} \PY{o}{=} \PY{n}{A}\PY{o}{.}\PY{n}{semidirect\PYZus{}product}\PY{p}{(}\PY{n}{B}\PY{p}{,}\PY{n}{hom1}\PY{p}{)} \PY{l+m+mi}{#Acción NO trivial}
\end{Verbatim}
\end{tcolorbox}

Por el Comentario \ref{2p}, hay dos posibles productos semidirectos que vendrán determinados por la acción tomada. 
Si la acción es la trivial, el producto semidirecto coincidirá con el producto directo y será el grupo cíclico $\mathbb{Z}_{2p}$.


    \begin{tcolorbox}[breakable, size=fbox, boxrule=1pt, pad at break*=1mm,colback=cellbackground, colframe=cellborder]
\prompt{In}{incolor}{10}{\boxspacing}
\begin{Verbatim}[commandchars=\\\{\}]
\PY{n}{A}\PY{o}{.}\PY{n}{direct\PYZus{}product}\PY{p}{(}\PY{n}{B}\PY{p}{)}\PY{o}{.}\PY{n}{is\PYZus{}isomorphic}\PY{p}{(}\PY{n}{SP0}\PY{p}{)}
\end{Verbatim}
\end{tcolorbox}

            \begin{tcolorbox}[breakable, size=fbox, boxrule=.5pt, pad at break*=1mm, opacityfill=0]
\prompt{Out}{outcolor}{10}{\boxspacing}
\begin{Verbatim}[commandchars=\\\{\}]
True
\end{Verbatim}
\end{tcolorbox}
        

Mientras que si consideramos la acción no trivial obtendremos un producto semidirecto que será isomorfo al grupo $D_p$. Para $n=3$, se tiene que $D_3 \cong S_3$ por lo que terminamos la construcción de $S_3$ como producto semidirecto viendo que $S_3 \cong A \rtimes_{hom1} B$.
\begin{tcolorbox}[breakable, size=fbox, boxrule=1pt, pad at break*=1mm,colback=cellbackground, colframe=cellborder]
\prompt{In}{incolor}{11}{\boxspacing}
\begin{Verbatim}[commandchars=\\\{\}]
\PY{n}{SP1}\PY{o}{.}\PY{n}{is\PYZus{}isomorphic}\PY{p}{(}\PY{n}{S}\PY{p}{)}
\end{Verbatim}
\end{tcolorbox}


\begin{tcolorbox}[breakable, size=fbox, boxrule=.5pt, pad at break*=1mm, opacityfill=0]
\prompt{Out}{outcolor}{11}{\boxspacing}
\begin{Verbatim}[commandchars=\\\{\}]
True
\end{Verbatim}
\end{tcolorbox}



\end{enumerate}

\iffalse
\newpage 
\begin{align*}
   G = \langle a,b,c,d \; | \; & a^2 = b^2 = c^2 = d^2 = 1,  \\
                            &(ab)^3 = (bc)^3 = (cd)^3 = 1, (ac)^2 = (bd)^2 = (ad)^2 = 1\rangle \quad y \quad H=\{1\}. 
\end{align*}



\begin{enumerate} 

    \begin{tcolorbox}[breakable, size=fbox, boxrule=1pt, pad at break*=1mm,colback=cellbackground, colframe=cellborder]
\prompt{In}{incolor}{35}{\boxspacing}
\begin{Verbatim}[commandchars=\\\{\}]
\PY{n}{file} \PY{o}{=} \PY{l+s+s2}{\PYZdq{}}\PY{l+s+s2}{Groups/S5.txt}\PY{l+s+s2}{\PYZdq{}}
\PY{n}{f} \PY{o}{=} \PY{n}{readGroup}\PY{p}{(}\PY{n}{file}\PY{p}{)}
\PY{n}{f}
\end{Verbatim}
\end{tcolorbox}

            \begin{tcolorbox}[breakable, size=fbox, boxrule=.5pt, pad at break*=1mm, opacityfill=0]
\prompt{Out}{outcolor}{35}{\boxspacing}
\begin{Verbatim}[commandchars=\\\{\}]
(['a', 'b', 'c', 'd'],
 ['aa', 'bb', 'cc', 'dd', 'ababab', 'bcbcbc', 'cdcdcd', 'acac', 'bdbd', 'adad'], [])
\end{Verbatim}
\end{tcolorbox}
        
    \begin{tcolorbox}[breakable, size=fbox, boxrule=1pt, pad at break*=1mm,colback=cellbackground, colframe=cellborder]
\prompt{In}{incolor}{36}{\boxspacing}
\begin{Verbatim}[commandchars=\\\{\}]
\PY{n}{G} \PY{o}{=} \PY{n}{CosetTable}\PY{p}{(}\PY{n}{f}\PY{p}{)}
\PY{n}{G}\PY{o}{.}\PY{n}{CosetEnumeration}\PY{p}{(}\PY{p}{)}
\end{Verbatim}
\end{tcolorbox}

    \begin{tcolorbox}[breakable, size=fbox, boxrule=1pt, pad at break*=1mm,colback=cellbackground, colframe=cellborder]
\prompt{In}{incolor}{37}{\boxspacing}
\begin{Verbatim}[commandchars=\\\{\}]
\PY{n}{generators} \PY{o}{=} \PY{n}{G}\PY{o}{.}\PY{n}{getGenerators}\PY{p}{(}\PY{p}{)}
\PY{n}{S} \PY{o}{=} \PY{n}{Group}\PY{p}{(}\PY{n}{elems} \PY{o}{=} \PY{n}{generators}\PY{p}{)}
\PY{n+nb}{print}\PY{p}{(}\PY{n}{S}\PY{p}{)}
\end{Verbatim}
\end{tcolorbox}

    \begin{Verbatim}[commandchars=\\\{\}]
Group with 120 elements.
    \end{Verbatim}

    \begin{tcolorbox}[breakable, size=fbox, boxrule=1pt, pad at break*=1mm,colback=cellbackground, colframe=cellborder]
\prompt{In}{incolor}{38}{\boxspacing}
\begin{Verbatim}[commandchars=\\\{\}]
\PY{n}{S}\PY{o}{.}\PY{n}{is\PYZus{}abelian}\PY{p}{(}\PY{p}{)}
\end{Verbatim}
\end{tcolorbox}

            \begin{tcolorbox}[breakable, size=fbox, boxrule=.5pt, pad at break*=1mm, opacityfill=0]
\prompt{Out}{outcolor}{38}{\boxspacing}
\begin{Verbatim}[commandchars=\\\{\}]
False
\end{Verbatim}
\end{tcolorbox}
        
    \begin{tcolorbox}[breakable, size=fbox, boxrule=1pt, pad at break*=1mm,colback=cellbackground, colframe=cellborder]
\prompt{In}{incolor}{39}{\boxspacing}
\begin{Verbatim}[commandchars=\\\{\}]
\PY{n}{S5} \PY{o}{=} \PY{n}{SymmetricGroup}\PY{p}{(}\PY{l+m+mi}{5}\PY{p}{)}
\PY{n}{S5}\PY{o}{.}\PY{n}{is\PYZus{}isomorphic}\PY{p}{(}\PY{n}{S}\PY{p}{)}
\end{Verbatim}
\end{tcolorbox}


            \begin{tcolorbox}[breakable, size=fbox, boxrule=.5pt, pad at break*=1mm, opacityfill=0]
\prompt{Out}{outcolor}{39}{\boxspacing}
\begin{Verbatim}[commandchars=\\\{\}]
True
\end{Verbatim}
\end{tcolorbox}
        
    

\end{enumerate}
\fi

\newpage

Consideramos el siguiente ejemplo, disponible en \textit{Groups/D4.txt}.
\begin{align*}
    G = \langle a,b \; | \; a^4, b^2, bab^{-1}a \rangle \quad  y \quad H=\langle b \rangle.
\end{align*}

\begin{enumerate} 

    \begin{tcolorbox}[breakable, size=fbox, boxrule=1pt, pad at break*=1mm,colback=cellbackground, colframe=cellborder]
\prompt{In}{incolor}{1}{\boxspacing}
\begin{Verbatim}[commandchars=\\\{\}]
\PY{n}{file} \PY{o}{=} \PY{l+s+s2}{\PYZdq{}}\PY{l+s+s2}{Groups/D4.txt}\PY{l+s+s2}{\PYZdq{}}
\PY{n}{f} \PY{o}{=} \PY{n}{readGroup}\PY{p}{(}\PY{n}{file}\PY{p}{)}
\PY{n}{f}
\end{Verbatim}
\end{tcolorbox}

    \begin{tcolorbox}[breakable, size=fbox, boxrule=.5pt, pad at break*=1mm, opacityfill=0]
\prompt{Out}{outcolor}{1}{\boxspacing}
\begin{Verbatim}[commandchars=\\\{\}]
(['a', 'b'], ['aaaa', 'bb', 'baBa'], ['b'])
\end{Verbatim}
\end{tcolorbox}
        
        
Por orden, ejecutamos el algoritmo y mostramos tanto la tabla de clases como el grafo de Schreier asociado.
    
    \begin{tcolorbox}[breakable, size=fbox, boxrule=1pt, pad at break*=1mm,colback=cellbackground, colframe=cellborder]
\prompt{In}{incolor}{2}{\boxspacing}
\begin{Verbatim}[commandchars=\\\{\}]
\PY{n}{G} \PY{o}{=} \PY{n}{CosetTable}\PY{p}{(}\PY{n}{f}\PY{p}{)}
\PY{n}{G}\PY{o}{.}\PY{n}{CosetEnumeration}\PY{p}{(}\PY{p}{)}
\end{Verbatim}
\end{tcolorbox}

    \begin{tcolorbox}[breakable, size=fbox, boxrule=1pt, pad at break*=1mm,colback=cellbackground, colframe=cellborder]
\prompt{In}{incolor}{3}{\boxspacing}
\begin{Verbatim}[commandchars=\\\{\}]
\PY{n+nb}{print}\PY{p}{(}\PY{n}{G}\PY{o}{.}\PY{n}{coset\PYZus{}table}\PY{p}{(}\PY{p}{)}\PY{p}{)}
\end{Verbatim}
\end{tcolorbox}

    \begin{center}
    \adjustimage{max size={0.28\linewidth}{0.28\paperheight}}{img/5g.png}
    \end{center}

    \begin{tcolorbox}[breakable, size=fbox, boxrule=1pt, pad at break*=1mm,colback=cellbackground, colframe=cellborder]
\prompt{In}{incolor}{4}{\boxspacing}
\begin{Verbatim}[commandchars=\\\{\}]
\PY{n}{G}\PY{o}{.}\PY{n}{schreier\PYZus{}graph}\PY{p}{(}\PY{n}{notes}\PY{o}{=}\PY{k+kc}{False}\PY{p}{)}
\end{Verbatim}
\end{tcolorbox}

    \begin{center}
    \adjustimage{max size={0.39\linewidth}{0.39\paperheight}}{img/code_69_0.png}
    \end{center}
    
    
Se tiene que:
\begin{align*}
   4 =  [G:H] = \frac{|G|}{|H|} = \frac{|G|}{2} \text{ , por lo que } |G|=8 \: .
\end{align*}
    
\iffalse
    \begin{tcolorbox}[breakable, size=fbox, boxrule=1pt, pad at break*=1mm,colback=cellbackground, colframe=cellborder]
\prompt{In}{incolor}{5}{\boxspacing}
\begin{Verbatim}[commandchars=\\\{\}]
\PY{n}{generators} \PY{o}{=} \PY{n}{G}\PY{o}{.}\PY{n}{getGenerators}\PY{p}{(}\PY{p}{)}
\PY{n}{print\PYZus{}gens}\PY{p}{(}\PY{n}{generators}\PY{p}{)}
\end{Verbatim}
\end{tcolorbox}


\begin{tcolorbox}[breakable, size=fbox, boxrule=.5pt, pad at break*=1mm, opacityfill=0]
\prompt{Out}{outcolor}{5}{\boxspacing}
    \begin{Verbatim}[commandchars=\\\{\}]
g0 = (1, 2, 3, 4)
g1 = (2, 4)
    \end{Verbatim}
    \end{tcolorbox}
\fi

\newpage
A continuación, construímos el grupo a partir de los generadores de Schreier y veremos que es isomorfo al grupo Diédrico $D_4$.
    \begin{tcolorbox}[breakable, size=fbox, boxrule=1pt, pad at break*=1mm,colback=cellbackground, colframe=cellborder]
\prompt{In}{incolor}{5}{\boxspacing}
\begin{Verbatim}[commandchars=\\\{\}]
\PY{n}{generators} \PY{o}{=} \PY{n}{G}\PY{o}{.}\PY{n}{getGenerators}\PY{p}{(}\PY{p}{)}
\PY{n}{S} \PY{o}{=} \PY{n}{Group}\PY{p}{(}\PY{n}{elems}\PY{o}{=}\PY{n}{generators}\PY{p}{)}
\PY{n+nb}{print}\PY{p}{(}\PY{n}{S}\PY{p}{)}
\end{Verbatim}
\end{tcolorbox}

            \begin{tcolorbox}[breakable, size=fbox, boxrule=.5pt, pad at break*=1mm, opacityfill=0]
\prompt{Out}{outcolor}{5}{\boxspacing}
    \begin{Verbatim}[commandchars=\\\{\}]
Group with 8 elements: \{(1, 2, 3, 4), (), (1, 2)(3, 4), (1, 4, 3, 2), (1, 4)(2, 3), (2, 4), (1, 3), (1, 3)(2, 4)\}
    \end{Verbatim}
    \end{tcolorbox}

    \begin{tcolorbox}[breakable, size=fbox, boxrule=1pt, pad at break*=1mm,colback=cellbackground, colframe=cellborder]
\prompt{In}{incolor}{6}{\boxspacing}
\begin{Verbatim}[commandchars=\\\{\}]
\PY{n}{D4} \PY{o}{=} \PY{n}{DihedralGroup}\PY{p}{(}\PY{l+m+mi}{4}\PY{p}{)}
\PY{n}{D4}\PY{o}{.}\PY{n}{is\PYZus{}isomorphic}\PY{p}{(}\PY{n}{S}\PY{p}{)}
\end{Verbatim}
\end{tcolorbox}

            \begin{tcolorbox}[breakable, size=fbox, boxrule=.5pt, pad at break*=1mm, opacityfill=0]
\prompt{Out}{outcolor}{6}{\boxspacing}
\begin{Verbatim}[commandchars=\\\{\}]
True
\end{Verbatim}
\end{tcolorbox}

El objetivo ahora es comprobar que el grupo $D_4$ es producto semidirecto interno de dos de sus subgrupos. Véase el Ejemplo \ref{dnrs}. En primer lugar, consideremos los subgrupos de $D_4$ generados por  $R1$  y  por $S0$:
    \begin{tcolorbox}[breakable, size=fbox, boxrule=1pt, pad at break*=1mm,colback=cellbackground, colframe=cellborder]
\prompt{In}{incolor}{7}{\boxspacing}
\begin{Verbatim}[commandchars=\\\{\}]
\PY{n}{R} \PY{o}{=} \PY{n}{D4}\PY{o}{.}\PY{n}{generate}\PY{p}{(}\PY{p}{[}\PY{l+s+s1}{\PYZsq{}}\PY{l+s+s1}{R1}\PY{l+s+s1}{\PYZsq{}}\PY{p}{]}\PY{p}{)}
\PY{n}{S} \PY{o}{=} \PY{n}{D4}\PY{o}{.}\PY{n}{generate}\PY{p}{(}\PY{p}{[}\PY{l+s+s1}{\PYZsq{}}\PY{l+s+s1}{S0}\PY{l+s+s1}{\PYZsq{}}\PY{p}{]}\PY{p}{)}
\PY{n+nb}{print}\PY{p}{(}\PY{n}{R, S}\PY{p}{)}
\end{Verbatim}
\end{tcolorbox}


            \begin{tcolorbox}[breakable, size=fbox, boxrule=.5pt, pad at break*=1mm, opacityfill=0]
\prompt{Out}{outcolor}{7}{\boxspacing}
    \begin{Verbatim}[commandchars=\\\{\}]
Group with 4 elements: \{'R3', 'R2', 'R0', 'R1'\}
Group with 2 elements: \{'S0', 'R0'\}
    \end{Verbatim}
    \end{tcolorbox}
    

Ahora bien, estudiamos las acciones $\varphi \colon \langle S0 \rangle \to \langle R1 \rangle$. Por el Comentario \ref{2p}, hay $2$ posibles productos semidirectos (uno de ellos  generado por la acción trivial). 





    \begin{tcolorbox}[breakable, size=fbox, boxrule=1pt, pad at break*=1mm,colback=cellbackground, colframe=cellborder]
\prompt{In}{incolor}{8}{\boxspacing}
\begin{Verbatim}[commandchars=\\\{\}]
\PY{n}{AutR}\PY{o}{ = }\PY{n}{R}\PY{o}{.}\PY{n}{AutomorphismGroup}\PY{p}{(}\PY{p}{)}
\PY{n}{Hom}\PY{o}{ = }\PY{n}{S}\PY{o}{.}\PY{n}{AllHomomorphisms}\PY{p}{(}\PY{n}{AutD}\PY{p}{)}
\end{Verbatim}
\end{tcolorbox}



    \begin{tcolorbox}[breakable, size=fbox, boxrule=1pt, pad at break*=1mm,colback=cellbackground, colframe=cellborder]
\prompt{In}{incolor}{9}{\boxspacing}
\begin{Verbatim}[commandchars=\\\{\}]
\PY{n}{hom0}\PY{o}{ = }\PY{n}{GroupHomomorphism}\PY{p}{(}\PY{n}{S}\PY{p}{,} \PY{n}{AutR}\PY{p}{,} \PY{k}{lambda} \PY{n}{x}\PY{p}{:}\PY{n}{Hom}\PY{p}{[}\PY{l+m+mi}{0}\PY{p}{]}\PY{p}{(}\PY{n}{x}\PY{p}{))}
\PY{n}{hom1}\PY{o}{ = }\PY{n}{GroupHomomorphism}\PY{p}{(}\PY{n}{S}\PY{p}{,} \PY{n}{AutR}\PY{p}{,} \PY{k}{lambda} \PY{n}{x}\PY{p}{:}\PY{n}{Hom}\PY{p}{[}\PY{l+m+mi}{1}\PY{p}{]}\PY{p}{(}\PY{n}{x}\PY{p}{))}

\PY{n}{SP0}\PY{o}{ = }\PY{n}{R}\PY{o}{.}\PY{n}{semidirect\PYZus{}product}\PY{p}{(}\PY{n}{S}\PY{p}{,}\PY{n}{hom0}\PY{p}{)} \PY{l+m+mi}{#Hom0 es la acción trivial.}
\PY{n}{SP1}\PY{o}{ = }\PY{n}{R}\PY{o}{.}\PY{n}{semidirect\PYZus{}product}\PY{p}{(}\PY{n}{S}\PY{p}{,}\PY{n}{hom1}\PY{p}{)} \PY{l+m+mi}{#Hom1 NO es trivial.}
\end{Verbatim}
\end{tcolorbox}


Bastará ver que el grupo Diédrico $D_4$ es producto semidirecto cuando la acción tomada no es la trivial, es decir:
\[
    D_4 \cong \langle R1 \rangle \rtimes_{hom1} \langle S0 \rangle \: .
\]
    \begin{tcolorbox}[breakable, size=fbox, boxrule=1pt, pad at break*=1mm,colback=cellbackground, colframe=cellborder]
\prompt{In}{incolor}{10}{\boxspacing}
\begin{Verbatim}[commandchars=\\\{\}]
\PY{n}{SP0}\PY{o}{.}\PY{n}{is\PYZus{}isomorphic}\PY{p}{(}\PY{n}{D4}\PY{p}{)}
\PY{n}{SP1}\PY{o}{.}\PY{n}{is\PYZus{}isomorphic}\PY{p}{(}\PY{n}{D4}\PY{p}{)}
\end{Verbatim}
\end{tcolorbox}




            \begin{tcolorbox}[breakable, size=fbox, boxrule=.5pt, pad at break*=1mm, opacityfill=0]
\prompt{Out}{outcolor}{11}{\boxspacing}
\begin{Verbatim}[commandchars=\\\{\}]
False, True
\end{Verbatim}
\end{tcolorbox}
    


\end{enumerate}



\iffalse
\newpage 
\begin{enumerate}


\begin{align*}
    G = \langle a,b \mid a^5, b^3, (ab)^2 \rangle \quad y \quad H=\{ 1 \}.
\end{align*}


\begin{tcolorbox}[breakable, size=fbox, boxrule=1pt, pad at break*=1mm,colback=cellbackground, colframe=cellborder]
\prompt{In}{incolor}{50}{\boxspacing}
\begin{Verbatim}[commandchars=\\\{\}]
\PY{n}{file} \PY{o}{=} \PY{l+s+s2}{\PYZdq{}}\PY{l+s+s2}{Groups/A5.txt}\PY{l+s+s2}{\PYZdq{}}
\PY{n}{f} \PY{o}{=} \PY{n}{readGroup}\PY{p}{(}\PY{n}{file}\PY{p}{)}
\PY{n}{f}
\end{Verbatim}
\end{tcolorbox}

            \begin{tcolorbox}[breakable, size=fbox, boxrule=.5pt, pad at break*=1mm, opacityfill=0]
\prompt{Out}{outcolor}{50}{\boxspacing}
\begin{Verbatim}[commandchars=\\\{\}]
(['a', 'b'], ['aaaaa', 'bbb', 'abab'], ['a'])
\end{Verbatim}
\end{tcolorbox}
        
    \begin{tcolorbox}[breakable, size=fbox, boxrule=1pt, pad at break*=1mm,colback=cellbackground, colframe=cellborder]
\prompt{In}{incolor}{51}{\boxspacing}
\begin{Verbatim}[commandchars=\\\{\}]
\PY{n}{G} \PY{o}{=} \PY{n}{CosetTable}\PY{p}{(}\PY{n}{f}\PY{p}{)}
\PY{n}{G}\PY{o}{.}\PY{n}{CosetEnumeration}\PY{p}{(}\PY{p}{)}
\end{Verbatim}
\end{tcolorbox}

    \begin{tcolorbox}[breakable, size=fbox, boxrule=1pt, pad at break*=1mm,colback=cellbackground, colframe=cellborder]
\prompt{In}{incolor}{52}{\boxspacing}
\begin{Verbatim}[commandchars=\\\{\}]
\PY{n}{generators} \PY{o}{=} \PY{n}{G}\PY{o}{.}\PY{n}{getGenerators}\PY{p}{(}\PY{p}{)}
\PY{n}{print\PYZus{}gens}\PY{p}{(}\PY{n}{generators}\PY{p}{)}
\end{Verbatim}
\end{tcolorbox}

    \begin{Verbatim}[commandchars=\\\{\}]
g0 = (2, 3, 4, 5, 6)(7, 9, 10, 11, 8)
g1 = (1, 2, 3)(4, 6, 7)(5, 8, 9)(10, 11, 12)
    \end{Verbatim}

    \begin{tcolorbox}[breakable, size=fbox, boxrule=1pt, pad at break*=1mm,colback=cellbackground, colframe=cellborder]
\prompt{In}{incolor}{53}{\boxspacing}
\begin{Verbatim}[commandchars=\\\{\}]
\PY{n}{S} \PY{o}{=} \PY{n}{Group}\PY{p}{(}\PY{n}{elems}\PY{o}{=}\PY{n}{generators}\PY{p}{)}
\PY{n}{S}\PY{o}{.}\PY{n}{order}\PY{p}{(}\PY{p}{)}
\end{Verbatim}
\end{tcolorbox}

            \begin{tcolorbox}[breakable, size=fbox, boxrule=.5pt, pad at break*=1mm, opacityfill=0]
\prompt{Out}{outcolor}{53}{\boxspacing}
\begin{Verbatim}[commandchars=\\\{\}]
60
\end{Verbatim}
\end{tcolorbox}
        
    \begin{tcolorbox}[breakable, size=fbox, boxrule=1pt, pad at break*=1mm,colback=cellbackground, colframe=cellborder]
\prompt{In}{incolor}{54}{\boxspacing}
\begin{Verbatim}[commandchars=\\\{\}]
\PY{n}{A5} \PY{o}{=} \PY{n}{AlternatingGroup}\PY{p}{(}\PY{l+m+mi}{5}\PY{p}{)}
\PY{n}{A5}\PY{o}{.}\PY{n}{is\PYZus{}isomorphic}\PY{p}{(}\PY{n}{S}\PY{p}{)}
                    
\end{Verbatim}
\end{tcolorbox}

            \begin{tcolorbox}[breakable, size=fbox, boxrule=.5pt, pad at break*=1mm, opacityfill=0]
\prompt{Out}{outcolor}{54}{\boxspacing}
\begin{Verbatim}[commandchars=\\\{\}]
True
\end{Verbatim}
\end{tcolorbox}
        
\end{enumerate}
\fi 



 
\begin{enumerate}
Por último, consideramos dos grupos $G$ y $H$ definidos como sigue:
\begin{align*}
    G = \langle a,b,c \mid a^6 = b^{2} = c^{2} = 1, abc \rangle  \quad y \quad 
H = \{ b\}.
\end{align*}

La definición del grupo se encuentra en el fichero \textit{Groups/3gens.txt}, y en principio no parece ser un grupo conocido.\\
Por orden: leemos el grupos haciendo uso del método \textit{readGroup}, llamamos al método que ejecuta el \textit{Algoritmo de Todd Coxeter}, y después vemos el número de clases que resultan en la tabla de clases de $G/H$ o el número de vértices del grafo de Schreier (no los mostramos por espacio), obteniendo que $[G:H]=6$.

El conjunto de generadores del grupo son:
\begin{align*}
    g0 &= (1, 2, 3, 4, 5, 6), \\
    g1 &= (2, 6)(3, 5), \\
    g2 &= (1, 6)(2, 5)(3, 4).
\end{align*}


Definimos el grupo $G= \langle g0, g1, g2 \rangle$, es decir, aquel generado por los elementos $g0, g1$ y $g2$, obteniendo:
    \begin{tcolorbox}[breakable, size=fbox, boxrule=1pt, pad at break*=1mm,colback=cellbackground, colframe=cellborder]
\prompt{In}{incolor}{1}{\boxspacing}
\begin{Verbatim}[commandchars=\\\{\}]
\PY{n+nb}{print}\PY{p}{(}\PY{n}{G}\PY{p}{)}
\end{Verbatim}
\end{tcolorbox}

    \begin{Verbatim}[commandchars=\\\{\}]
Group with 12 elements: \{(1, 5)(2, 4), (1, 2, 3, 4, 5, 6), (1, 4)(2, 5)(3, 6), (1, 6)(2, 5)(3, 4), (1, 3)(4, 6), (1, 4)(2, 3)(5, 6), (), (2, 6)(3, 5), (1, 2)(3, 6)(4, 5), (1, 6, 5, 4, 3, 2), (1, 3, 5)(2, 4, 6), (1, 5, 3)(2, 6, 4)\}
    \end{Verbatim}

    \begin{tcolorbox}[breakable, size=fbox, boxrule=1pt, pad at break*=1mm,colback=cellbackground, colframe=cellborder]
\prompt{In}{incolor}{2}{\boxspacing}
\begin{Verbatim}[commandchars=\\\{\}]
\PY{n}{group}\PY{o}{.}\PY{n}{is\PYZus{}abelian}\PY{p}{(}\PY{p}{)}
\end{Verbatim}
\end{tcolorbox}

            \begin{tcolorbox}[breakable, size=fbox, boxrule=.5pt, pad at break*=1mm, opacityfill=0]
\prompt{Out}{outcolor}{2}{\boxspacing}
\begin{Verbatim}[commandchars=\\\{\}]
False
\end{Verbatim}
\end{tcolorbox}
        
    El grupo no es abeliano, luego debe ser isomorfo a uno de los siguientes grupos:
\begin{align*}
    &G \cong A_4 = \{ a,b \; | \; a^3=b^3=(ab)^2=1 \} , \\
    &G \cong D_6 = \{ a,b \; | \; a^6=b^2=1, ab=a^{-1}b \}, \\ 
    &G \cong Q_3 = \{ a,b \; | \; a^{6}=1, a^3=b^2, ab=a^{-1}b \}.
\end{align*}

    \begin{tcolorbox}[breakable, size=fbox, boxrule=1pt, pad at break*=1mm,colback=cellbackground, colframe=cellborder]
\prompt{In}{incolor}{3}{\boxspacing}
\begin{Verbatim}[commandchars=\\\{\}]
\PY{n}{A} \PY{o}{=} \PY{n}{AlternatingGroup}\PY{p}{(}\PY{l+m+mi}{4}\PY{p}{)}
\PY{n}{D} \PY{o}{=} \PY{n}{DihedralGroup}\PY{p}{(}\PY{l+m+mi}{6}\PY{p}{)}
\PY{n}{Q} \PY{o}{=} \PY{n}{QuaternionGroupGeneralised}\PY{p}{(}\PY{l+m+mi}{3}\PY{p}{)}

\PY{n+nb}{print}\PY{p}{(}\PY{n}{group}\PY{o}{.}\PY{n}{is\PYZus{}isomorphic}\PY{p}{(}\PY{n}{A}\PY{p}{)}\PY{p}{)}
\PY{n+nb}{print}\PY{p}{(}\PY{n}{group}\PY{o}{.}\PY{n}{is\PYZus{}isomorphic}\PY{p}{(}\PY{n}{D}\PY{p}{)}\PY{p}{)}
\PY{n+nb}{print}\PY{p}{(}\PY{n}{group}\PY{o}{.}\PY{n}{is\PYZus{}isomorphic}\PY{p}{(}\PY{n}{Q}\PY{p}{)}\PY{p}{)}
\end{Verbatim}
\end{tcolorbox}

 \begin{tcolorbox}[breakable, size=fbox, boxrule=.5pt, pad at break*=1mm, opacityfill=0]
\prompt{Out}{outcolor}{3}{\boxspacing}
    \begin{Verbatim}[commandchars=\\\{\}]
False, True, False
    \end{Verbatim}
\end{tcolorbox}
Como hemos visto, el grupo $G$ se trata del grupo Diédrico $D_6$. Como consecuencia, vemos que la presentación de un grupo no es única.
\end{enumerate}



\newpage
\subsection{Otras presentaciones}
Para acabar, consideraremos grupos finitamente presentados que no se han estudiado en este proyecto. El objetivo es el de mostrar la potencia que tiene este método programado, llegando incluso a terminar en poco tiempo con ejemplos de grupos de orden muy alto.

\begin{enumerate}
\item Consideramos el grupo especial lineal $SL_n(F) \colon = \{ A \in M_n(F) \: | \: det(A)=1 \}$ y su centro $Z(SL_n(F))$. Definimos el grupo lineal especial proyectivo $PSL_n(F)$ como el cociente entre $SL_n(F)$ por $Z(SL_2(F))$.

Consideramos un campo finito con $7$ elementos, por ejemplo $\mathbb{Z}_7$, entonces $PSL_2(\mathbb{Z}_7)$ viene definido por:
\begin{align*}
        G = \langle a,b,c \mid a^7 = b^3 = c^2 = 1, ba=aab, (bc)^2, (ac)^2  \rangle  .
\end{align*}


Este grupo se encuentra definido en el fichero \textit{Groups/PSL2.txt} y se tomará el grupo trivial $H$ para su ejecución. A pesar de ser un grupo de orden $168$, termina en torno a $15$ segundos con el método \textit{HLT}.


        
\item Sea $G$ un grupo generado por tres elementos $a,b$ y $c$ que satisfacen las siguientes relaciones:
\begin{align*} \label{ye1}
    a^3=b^2=c^2=1,\: (ab)^4=(ac)^2=(bc)^3=1 \:.
\end{align*}
y $H \leq G$ generado por $a$ y $b$, es decir:
\begin{align*}
    G = \langle a,b,c \mid a^3=b^2=c^2=1,\: (ab)^4=(ac)^2=(bc)^3=1 \rangle \quad y \quad H=\langle a,b \rangle.
    %G = \langle a,b,c \; | \; b^2c^{-1}bc, a^2b^{-1}ab, cab^{-1}cabc \rangle \quad y \quad H=\langle a,b \rangle.
\end{align*}
Se encuentra disponible en \textit{Group/G0.txt} y se trata de un grupo de orden $576$. Su ejecución con el método programado tarda en torno a $1$ minuto. De igual modo, es costoso generar todo el grupo de Permutaciones y realizar operaciones con los métodos de la librería.



\item El siguiente grupo es conocido como grupo de Mathieu,  descubierto a finales del siglo $XIX$ por el matemático francés Émile Mathieu junto a otros cuatro grupos de permutaciones. Se denota por $M_{12}$ y viene dado por:
\begin{align*}
    M_{12} = \langle a,b,c \mid a^{11} = b^2 = c^2 = 1, (ab)^3 = (ac)^3 = (bc)^{10} = 1, a^2(bc)^2a = (bc)^2  \rangle.
\end{align*}

Se encuentra definido en el archivo \textit{Groups/Big.txt}, y como subgrupo $H$, se ha tomado el trivial. Tras aplicar el \textit{Algoritmo de Todd Coxeter}, obtenemos que el índice $[G:H]=|G|=95040$. 
 Requiere en torno a $600$.$000$ clases laterales y su ejecución con el método \textit{HLT} se demora en torno a $5$ minutos. Por esta razón, no es pensable obtener los generadores de Schreier ni definir el grupo con estos ya que cualquier operación básica que se realice consumiría mucho tiempo.
 
\end{enumerate}
    