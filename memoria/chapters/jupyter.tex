

En esta sección se comentarán los cambios realizados en cada
clase de la librería, que se han implementado siguiendo los conceptos matemáticos descritos anteriormente. Además, se ha completado y añadido la documentación de cada método y función implementada, por lo que el usuario puede consultarla si así lo desea. \\
Se usará Jupyter para ilustrar algunos ejemplos, donde únicamente bastará con importar la librería que queramos para poder usar todos sus métodos implementados.

\begin{itemize}

\item \texttt{Set.py}: se han añadido métodos para realizar operaciones
  a nivel de conjunto.

  \begin{itemize}
  \item Unión, diferencia, intersección, producto cartesiano y diferencia
    simétrica.


    \begin{tcolorbox}[breakable, size=fbox, boxrule=1pt, pad at break*=1mm,colback=cellbackground, colframe=cellborder]
\prompt{In}{incolor}{1}{\boxspacing}
\begin{Verbatim}[commandchars=\\\{\}]
\PY{k+kn}{from} \PY{n+nn}{Set} \PY{k+kn}{import} \PY{n}{Set}
\end{Verbatim}
\end{tcolorbox}

    
    \begin{tcolorbox}[breakable, size=fbox, boxrule=1pt, pad at break*=1mm,colback=cellbackground, colframe=cellborder]
\prompt{In}{incolor}{2}{\boxspacing}
\begin{Verbatim}[commandchars=\\\{\}]
\PY{n}{A} \PY{o}{=} \PY{n}{Set}\PY{p}{(}\PY{p}{\PYZob{}}\PY{l+m+mi}{1}\PY{p}{,}\PY{l+m+mi}{2}\PY{p}{,}\PY{l+m+mi}{3}\PY{p}{\PYZcb{}}\PY{p}{)}
\PY{n}{B} \PY{o}{=} \PY{n}{Set}\PY{p}{(}\PY{p}{\PYZob{}}\PY{l+m+mi}{2}\PY{p}{,}\PY{l+m+mi}{4}\PY{p}{\PYZcb{}}\PY{p}{)}
\end{Verbatim}
\end{tcolorbox}


    \begin{tcolorbox}[breakable, size=fbox, boxrule=1pt, pad at break*=1mm,colback=cellbackground, colframe=cellborder]
\prompt{In}{incolor}{3}{\boxspacing}
\begin{Verbatim}[commandchars=\\\{\}]
\PY{n}{C} \PY{o}{=} \PY{n}{A}\PY{o}{*}\PY{n}{B}
\PY{n}{C}
\end{Verbatim}
\end{tcolorbox}


    \begin{tcolorbox}[breakable, size=fbox, boxrule=.5pt, pad at break*=1mm, opacityfill=0]
\prompt{Out}{outcolor}{3}{\boxspacing}
\begin{Verbatim}[commandchars=\\\{\}]
\{(2, 4), (1, 2), (3, 4), (2, 2), (3, 2), (1, 4)\}
\end{Verbatim}
\end{tcolorbox}
        

  

    \item \textit{cardinality}, \textit{is\_finite}: se tratan de métodos que sirven para calcular la cardinalidad del conjunto y comprobar si este es finito, respectivamente.


    \begin{tcolorbox}[breakable, size=fbox, boxrule=1pt, pad at break*=1mm,colback=cellbackground, colframe=cellborder]
\prompt{In}{incolor}{4}{\boxspacing}
\begin{Verbatim}[commandchars=\\\{\}]
\PY{n}{C}\PY{o}{.}\PY{n}{is\PYZus{}finite}\PY{p}{(}\PY{p}{)}\PY{p}{,} \PY{n}{C}\PY{o}{.}\PY{n}{cardinality}\PY{p}{(}\PY{p}{)}
\end{Verbatim}
\end{tcolorbox}

    \begin{tcolorbox}[breakable, size=fbox, boxrule=.5pt, pad at break*=1mm, opacityfill=0]
\prompt{Out}{outcolor}{4}{\boxspacing}
\begin{Verbatim}[commandchars=\\\{\}]
(True, 6)
\end{Verbatim}
\end{tcolorbox}
        
        
    \item \textit{subsets}:  este método se encarga de calcular los subconjuntos de un conjunto. Si por parámetro se le pasa un número natural $n$, entonces calculará los subconjuntos de tamaño $n$.
    


    \begin{tcolorbox}[breakable, size=fbox, boxrule=1pt, pad at break*=1mm,colback=cellbackground, colframe=cellborder]
\prompt{In}{incolor}{5}{\boxspacing}
\begin{Verbatim}[commandchars=\\\{\}]
\PY{n}{A}\PY{o}{.}\PY{n}{subsets}\PY{p}{(}\PY{p}{)}
\end{Verbatim}
\end{tcolorbox}


\begin{tcolorbox}[breakable, size=fbox, boxrule=.5pt, pad at break*=1mm, opacityfill=0]
\prompt{Out}{outcolor}{5}{\boxspacing}
\begin{Verbatim}[commandchars=\\\{\}]
[\{1\}, \{2\}, \{3\}, \{1, 2\}, \{1, 3\}, \{2, 3\}, \{1, 2, 3\}]
\end{Verbatim}
\end{tcolorbox}
        
    \begin{tcolorbox}[breakable, size=fbox, boxrule=1pt, pad at break*=1mm,colback=cellbackground, colframe=cellborder]
\prompt{In}{incolor}{6}{\boxspacing}
\begin{Verbatim}[commandchars=\\\{\}]
\PY{n}{A}\PY{o}{.}\PY{n}{subsets}\PY{p}{(}\PY{l+m+mi}{2}\PY{p}{)}
\end{Verbatim}
\end{tcolorbox}

        \begin{tcolorbox}[breakable, size=fbox, boxrule=.5pt, pad at break*=1mm, opacityfill=0]
\prompt{Out}{outcolor}{6}{\boxspacing}
\begin{Verbatim}[commandchars=\\\{\}]
[\{1, 2\}, \{1, 3\}, \{2, 3\}]
\end{Verbatim}
\end{tcolorbox}
        
        
    \end{itemize}
  
  
\newpage
\item  \texttt{Function.py}: se ha mantenido en tu totalidad el formato
  original, a excepción del operador \textit{\_\_str\_\_} que muestra ahora
  la función de manera clara y precisa.


    \begin{tcolorbox}[breakable, size=fbox, boxrule=1pt, pad at break*=1mm,colback=cellbackground, colframe=cellborder]
\prompt{In}{incolor}{7}{\boxspacing}
\begin{Verbatim}[commandchars=\\\{\}]
\PY{k+kn}{from} \PY{n+nn}{Function} \PY{k+kn}{import} \PY{n}{Function}
\end{Verbatim}
\end{tcolorbox}

    Sea S un conjunto de tipo Set, para definir, por ejemplo, la siguiente operación binaria
    \begin{align*}
        S \times S &\rightarrow S \\
            (x,y) & \mapsto (x+y)\%3 
    \end{align*}

usaremos las función lambda que nos ofrece Python.

    \begin{tcolorbox}[breakable, size=fbox, boxrule=1pt, pad at break*=1mm,colback=cellbackground, colframe=cellborder]
\prompt{In}{incolor}{8}{\boxspacing}
\begin{Verbatim}[commandchars=\\\{\}]
\PY{n}{S} \PY{o}{=} \PY{n}{Set}\PY{p}{(}\PY{p}{\PYZob{}}\PY{l+m+mi}{0}\PY{p}{,}\PY{l+m+mi}{1}\PY{p}{,}\PY{l+m+mi}{2}\PY{p}{\PYZcb{}}\PY{p}{)}
\PY{n}{F} \PY{o}{=} \PY{n}{Function}\PY{p}{(}\PY{n}{S}\PY{o}{*}\PY{n}{S}\PY{p}{,} \PY{n}{S}\PY{p}{,}\PY{k}{lambda} \PY{n}{x}\PY{p}{:} \PY{p}{(}\PY{n}{x}\PY{p}{[}\PY{l+m+mi}{0}\PY{p}{]}\PY{o}{+}\PY{n}{x}\PY{p}{[}\PY{l+m+mi}{1}\PY{p}{]}\PY{p}{)}\PY{o}{\PYZpc{}}\PY{k}{3})
\PY{n+nb}{print}\PY{p}{(}\PY{n}{F}\PY{p}{)}
\end{Verbatim}
\end{tcolorbox}

    \begin{Verbatim}[commandchars=\\\{\}]
    f((0, 1))=1
    f((1, 2))=0
    f((2, 1))=0
    f((0, 0))=0
    f((1, 1))=2
    f((2, 0))=2
    f((0, 2))=2
    f((2, 2))=1
    f((1, 0))=1
    \end{Verbatim}

    

\item  \texttt{Group.py}:

\begin{itemize}
    \item \textit{\_\_str\_\_} y \textit{\_\_repr\_\_}: se modifican
      para además mostrar los elementos del grupo (siempre que el orden del
      grupo no sea grande).
    \item Se ha modificado el constructor \textit{\_\_init\_\_} de la clase
      \textit{Group}. De este modo, se podrán definir grupos de las dos
      formas comentadas en la Sección \ref{pg}.

  \begin{enumerate}

  \item Definición axiomatixada. Se comprueba que el par
    \textit{(Set, Function)} pasado por argumento satisface los axiomas de grupo (asociatividad, identidad e inversos).


    \begin{tcolorbox}[breakable, size=fbox, boxrule=1pt, pad at break*=1mm,colback=cellbackground, colframe=cellborder]
\prompt{In}{incolor}{9}{\boxspacing}
\begin{Verbatim}[commandchars=\\\{\}]
\PY{k+kn}{from} \PY{n+nn}{Group} \PY{k+kn}{import} \PY{o}{*}
\end{Verbatim}
\end{tcolorbox}

    
    \begin{tcolorbox}[breakable, size=fbox, boxrule=1pt, pad at break*=1mm,colback=cellbackground, colframe=cellborder]
\prompt{In}{incolor}{10}{\boxspacing}
\begin{Verbatim}[commandchars=\\\{\}]
\PY{n}{S} \PY{o}{=} \PY{n}{Set}\PY{p}{(}\PY{p}{\PYZob{}}\PY{l+m+mi}{0}\PY{p}{,}\PY{l+m+mi}{1}\PY{p}{,}\PY{l+m+mi}{2}\PY{p}{,}\PY{l+m+mi}{3}\PY{p}{\PYZcb{}}\PY{p}{)}
\PY{n}{F}\PY{o} {=} \PY{n}{Function}\PY{p}{(}\PY{n}{S}\PY{o}{*}\PY{n}{S}\PY{p}{,} \PY{n}{S}\PY{p}{,}\PY{k}{lambda} \PY{n}{x}\PY{p}{:} \PY{p}{(}\PY{n}{x}\PY{p}{[}\PY{l+m+mi}{0}\PY{p}{]}\PY{o}{+}\PY{n}{x}\PY{p}{[}\PY{l+m+mi}{1}\PY{p}{]}\PY{p}{)}\PY{o}{\PYZpc{}}\PY{k}{4})
\PY{n}{Z4} \PY{o}{=} \PY{n}{Group}\PY{p}{(}\PY{n}{S}\PY{p}{,}\PY{n}{F}\PY{p}{)}

\PY{n+nb}{print}\PY{p}{(}\PY{n}{Z4}\PY{p}{)}
\end{Verbatim}
\end{tcolorbox}

\begin{tcolorbox}[breakable, size=fbox, boxrule=.5pt, pad at break*=1mm, opacityfill=0]
\prompt{Out}{outcolor}{10}{\boxspacing}
    \begin{Verbatim}[commandchars=\\\{\}]
Group with 4 elements: \{0, 1, 2, 3\}
    \end{Verbatim}
\end{tcolorbox}

\item Definición en términos de generadores y relatores. Sea un grupo $G = \langle X \mid R\rangle$. Se pasa por argumento el conjunto de generadores $X$ y relaciones $R$ que definen al grupo. El constructor se encarga de aplicar el \textit{Algoritmo de Todd Coxeter} y darle estructura de grupo de Permutaciones al grupo $G$. Se tomará el subgrupo trivial para esta ejecución del algoritmo.


A continuación definimos el grupo $G= \langle a \mid a^4 =1 \rangle$.
    \begin{tcolorbox}[breakable, size=fbox, boxrule=1pt, pad at break*=1mm,colback=cellbackground, colframe=cellborder]
\prompt{In}{incolor}{11}{\boxspacing}
\begin{Verbatim}[commandchars=\\\{\}]
\PY{n}{gens} \PY{o}{=} \PY{p}{[}\PY{l+s+s1}{\PYZsq{}}\PY{l+s+s1}{a}\PY{l+s+s1}{\PYZsq{}}\PY{p}{]}
\PY{n}{rels} \PY{o}{=} \PY{p}{[}\PY{l+s+s1}{\PYZsq{}}\PY{l+s+s1}{aaaa}\PY{l+s+s1}{\PYZsq{}}\PY{p}{]} \PY{l+m+mi}{#a^4=1}

\PY{n}{G} \PY{o}{=} \PY{n}{Group}\PY{p}{(}\PY{n}{gensG}\PY{o}{=}\PY{n}{gens}\PY{p}{,} \PY{n}{relsG}\PY{o}{=}\PY{n}{rels}\PY{p}{)}
\PY{n+nb}{print}\PY{p}{(}\PY{n}{G}\PY{p}{)}
\end{Verbatim}
\end{tcolorbox}

    \begin{Verbatim}[commandchars=\\\{\}]
Group with 4 elements: \{(), (1, 2, 3, 4), (1, 4, 3, 2), (1, 3)(2, 4)\}
    \end{Verbatim}

Naturalmente, y aunque la forma de definir ambos grupos anteriores es distinta, son isomorfos; es
decir, $G=\langle a \mid a^4=1 \rangle \cong \mathbb{Z}_4$.

    \begin{tcolorbox}[breakable, size=fbox, boxrule=1pt, pad at break*=1mm,colback=cellbackground, colframe=cellborder]
\prompt{In}{incolor}{12}{\boxspacing}
\begin{Verbatim}[commandchars=\\\{\}]
\PY{n}{G}\PY{o}{.}\PY{n}{is\PYZus{}isomorphic}\PY{p}{(}\PY{n}{Z4}\PY{p}{)}
\end{Verbatim}
\end{tcolorbox}

\begin{tcolorbox}[breakable, size=fbox, boxrule=.5pt, pad at break*=1mm, opacityfill=0]
\prompt{Out}{outcolor}{12}{\boxspacing}
\begin{Verbatim}[commandchars=\\\{\}]
True
\end{Verbatim}
\end{tcolorbox}



Por último, se añadirá una tercera forma de definir un grupo. Sea $Y$ un conjunto de elementos, entonces el grupo $G$ se definirá como el grupo generado por $\langle Y \rangle$. En el siguiente ejemplo tomaremos un conjunto con una única permutación, sin embargo, no exigimos que los elementos sean permutaciones.
    \begin{tcolorbox}[breakable, size=fbox, boxrule=1pt, pad at break*=1mm,colback=cellbackground, colframe=cellborder]
\prompt{In}{incolor}{11}{\boxspacing}
\begin{Verbatim}[commandchars=\\\{\}]
\PY{n}{p} \PY{o}{=} \PY{n}{permutation}\PY{p}{(}\PY{p}{(}\PY{l+m+mi}{1}\PY{p}{,}\PY{l+m+mi}{2}\PY{p}{,}\PY{l+m+mi}{3}\PY{p}{,}\PY{l+m+mi}{4}\PY{p}{)}\PY{p}{)}
\PY{n}{G} \PY{o}{=} \PY{n}{Group}\PY{p}{(}\PY{n}{elems}\PY{o}{=}\PY{p}{[}\PY{n}{p}\PY{p}{]}\PY{p}{)}
\PY{n+nb}{print}\PY{p}{(}\PY{n}{G}\PY{p}{)}
\end{Verbatim}
\end{tcolorbox}

    \begin{Verbatim}[commandchars=\\\{\}]
Group with 4 elements: \{(), (1, 2, 3, 4), (1, 4, 3, 2), (1, 3)(2, 4)\}
    \end{Verbatim}


  \end{enumerate}


        
   \item  \textit{is\_abelian}: en una primera versión se comprobaba si el
grupo era abeliano en el constructor y se hacía uso de una variable de
clase. Se añade este método para realizar esta comprobación.

 \item \textit{identity}: del mismo modo que en \textit{is\_abelian},
se añade un nuevo método para calcular la identidad del grupo.

    \begin{tcolorbox}[breakable, size=fbox, boxrule=1pt, pad at break*=1mm,colback=cellbackground, colframe=cellborder]
\prompt{In}{incolor}{13}{\boxspacing}
\begin{Verbatim}[commandchars=\\\{\}]
\PY{n}{Z4}\PY{o}{.}\PY{n}{is\PYZus{}abelian}\PY{p}{(}\PY{p}{)}
\end{Verbatim}
\end{tcolorbox}

            \begin{tcolorbox}[breakable, size=fbox, boxrule=.5pt, pad at break*=1mm, opacityfill=0]
\prompt{Out}{outcolor}{13}{\boxspacing}
\begin{Verbatim}[commandchars=\\\{\}]
True
\end{Verbatim}
\end{tcolorbox}



\newpage 

    \begin{tcolorbox}[breakable, size=fbox, boxrule=1pt, pad at break*=1mm,colback=cellbackground, colframe=cellborder]
\prompt{In}{incolor}{14}{\boxspacing}
\begin{Verbatim}[commandchars=\\\{\}]
\PY{n}{Z4}\PY{o}{.}\PY{n}{identity}\PY{p}{(}\PY{p}{)} \PY{p}{,} \PY{n}{G}\PY{o}{.}\PY{n}{identity}\PY{p}{(}\PY{p}{)}
\end{Verbatim}
\end{tcolorbox}

            \begin{tcolorbox}[breakable, size=fbox, boxrule=.5pt, pad at break*=1mm, opacityfill=0]
\prompt{Out}{outcolor}{14}{\boxspacing}
\begin{Verbatim}[commandchars=\\\{\}]
(0, ())
\end{Verbatim}
\end{tcolorbox}
        
   \item  \textit{cosets}: método que calcula las clases laterales de un grupo
$G$ sobre un subgrupo $H$. Se optimiza y se simplifica.

\end{itemize}

\item   \texttt{Permutation.py}: la clase \textit{Permutation} es la
  que se encarga de construir permutaciones y da lugar al grupo
  Simétrico y Alternado. En esta clase no se han realizado importantes
  modificaciones, sin embargo, requiere de una mención especial ya que
  basándonos en el Teorema \ref{important}, se programará un
  método que le proporcionará una estructura de grupo de Permutaciones a
  cualquier grupo, en especial, a los grupos definidos por una
  presentación.

  Las modificaciones realizadas han sido las siguientes:
  \begin{itemize}
    \item \textit{\_\_mul\_\_}: se modifica el operador encargado de
  multiplicar dos permutaciones. Optimización y simplificación del
  código.
    \item \textit{\_\_call\_\_}: este operador se encarga de calcular la
  imagen de un elemento de una permutación. Arrojaba un error de
  compilación que ya se ha corregido.


    \begin{tcolorbox}[breakable, size=fbox, boxrule=1pt, pad at break*=1mm,colback=cellbackground, colframe=cellborder]
\prompt{In}{incolor}{15}{\boxspacing}
\begin{Verbatim}[commandchars=\\\{\}]
\PY{n}{p} \PY{o}{=} \PY{n}{permutation}\PY{p}{(}\PY{p}{(}\PY{l+m+mi}{1}\PY{p}{,}\PY{l+m+mi}{3}\PY{p}{)}\PY{p}{,}\PY{p}{(}\PY{l+m+mi}{5}\PY{p}{,}\PY{l+m+mi}{2}\PY{p}{)}\PY{p}{)}

\PY{k}{for} \PY{n}{i} \PY{o+ow}{in} \PY{n+nb}{range}\PY{p}{(}\PY{l+m+mi}{1}\PY{p}{,}\PY{l+m+mi}{6}\PY{p}{)}\PY{p}{:}
    \PY{n+nb}{print}\PY{p}{(}\PY{l+s+s2}{\PYZdq{}}\PY{l+s+s2}{p(}\PY{l+s+si}{\PYZob{}\PYZcb{}}\PY{l+s+s2}{)=}\PY{l+s+si}{\PYZob{}\PYZcb{}}\PY{l+s+s2}{\PYZdq{}}\PY{o}{.}\PY{n}{format}\PY{p}{(}\PY{n}{i}\PY{p}{,} \PY{n}{p}\PY{p}{(}\PY{n}{i}\PY{p}{)}\PY{p}{)}\PY{p}{)}
\end{Verbatim}
\end{tcolorbox}

    \begin{Verbatim}[commandchars=\\\{\}]
p(1)=3
p(2)=5
p(3)=1
p(4)=4
p(5)=2
    \end{Verbatim}

   \item  \textit{even\_permutation}, \textit{odd\_permutation}: se añaden los
siguientes métodos encargados de calcular si una permutación es par o
impar.

    \begin{tcolorbox}[breakable, size=fbox, boxrule=1pt, pad at break*=1mm,colback=cellbackground, colframe=cellborder]
\prompt{In}{incolor}{16}{\boxspacing}
\begin{Verbatim}[commandchars=\\\{\}]
\PY{n}{p}\PY{o}{.}\PY{n}{odd\PYZus{}permutation}\PY{p}{(}\PY{p}{)}
\end{Verbatim}
\end{tcolorbox}

            \begin{tcolorbox}[breakable, size=fbox, boxrule=.5pt, pad at break*=1mm, opacityfill=0]
\prompt{Out}{outcolor}{16}{\boxspacing}
\begin{Verbatim}[commandchars=\\\{\}]
False
\end{Verbatim}
\end{tcolorbox}

\end{itemize}

\item \texttt{Complex.py}: se ha realizado una implementación de la
  clase número complejo, \textit{class Complex}, junto a todos los
  operadores necesarios para realizar operaciones entre números
  complejos. Gracias a esta clase, se programa el grupo de las raíces
  n-ésimas de la unidad y una función que se encarga de representar sus
  soluciones:
  \begin{itemize}
    \item \textit{plot(G, rep)}: dado un grupo de las raíces n-ésimas de
  la unidad pasado por argumento, esta función representa todas sus
  raíces en el plano complejo. El segundo parámetro \textit{rep} permite elegir el modo de representación, que puede ser ``exp'' (por defecto) para mostrarlos mediante la representación exponencial o ``binom'' para mostrarlos usando su forma binomial $a+bi$.


    \begin{tcolorbox}[breakable, size=fbox, boxrule=1pt, pad at break*=1mm,colback=cellbackground, colframe=cellborder]
\prompt{In}{incolor}{17}{\boxspacing}
\begin{Verbatim}[commandchars=\\\{\}]
\PY{n}{G} \PY{o}{=} \PY{n}{RootsOfUnitGroup}\PY{p}{(}\PY{l+m+mi}{5}\PY{p}{)}
\PY{n}{plot}\PY{p}{(}\PY{n}{G}\PY{p}{)}
\end{Verbatim}
\end{tcolorbox}

    %\begin{center}
    %\adjustimage{max size={1.5\linewidth}{1.5\paperheight}}{img/raiz.png}
    %\end{center}

    \begin{center}
    \adjustimage{max size={0.29\linewidth}{0.29\paperheight}}{img/1g.png}
    \end{center}




    \end{itemize}

\item \texttt{Quaternion.py}: como hemos comentado anteriormente, uno de
  los problemas que tenía la librería  y que se quería corregir era evitar tener que dar la tabla de multiplicar de un grupo.
  Por ello, se realiza una implementación de los números cuaternios en
  la clase \textit{Quaternion} sobrecargando el operador
  \textit{\_\_mul\_\_} para dotar a estos números de su producto.
  
  \begin{itemize}
        \item Se han programado todos los operadores necesarios para trabajar y
      operar con números cuaternios, desde su manejo y representación
      \textit{\_\_repr\_\_}, \textit{\_\_str\_\_}, \textit{\_\_call\_\_},
      hasta los operadores encargados de sumar, restar, multiplicar, dividir
      (\textit{\_\_add\_\_}, \textit{\_\_sub\_\_}, \textit{\_\_mull\_\_},
      \textit{\_\_div\_\_})\ldots etc.
        
     \item Se han implementado métodos como \textit{conjugate}, \textit{norm},
      \textit{inverse}, \textit{trace}, encargados de calcular el conjugado,
      norma, inverso y traza, respectivamente.



    \begin{tcolorbox}[breakable, size=fbox, boxrule=1pt, pad at break*=1mm,colback=cellbackground, colframe=cellborder]
\prompt{In}{incolor}{18}{\boxspacing}
\begin{Verbatim}[commandchars=\\\{\}]
\PY{n}{q} \PY{o}{=} \PY{n}{Quaternion}\PY{p}{(}\PY{o}{\PYZhy{}}\PY{l+m+mi}{3}\PY{p}{,}\PY{l+m+mi}{1}\PY{p}{,}\PY{l+m+mi}{2}\PY{p}{,}\PY{o}{\PYZhy{}}\PY{l+m+mi}{8}\PY{p}{)}
\PY{n}{p} \PY{o}{=} \PY{n}{Quaternion}\PY{p}{(}\PY{l+m+mi}{0}\PY{p}{,}\PY{l+m+mi}{2}\PY{p}{,}\PY{l+m+mi}{3}\PY{p}{,}\PY{l+m+mi}{1}\PY{p}{)}
\PY{n}{q}\PY{o}{+}\PY{n}{p}
\end{Verbatim}
\end{tcolorbox}

            \begin{tcolorbox}[breakable, size=fbox, boxrule=.5pt, pad at break*=1mm, opacityfill=0]
\prompt{Out}{outcolor}{18}{\boxspacing}
\begin{Verbatim}[commandchars=\\\{\}]
 -3+3i+5j-7k
\end{Verbatim}
\end{tcolorbox}
        
    \begin{tcolorbox}[breakable, size=fbox, boxrule=1pt, pad at break*=1mm,colback=cellbackground, colframe=cellborder]
\prompt{In}{incolor}{19}{\boxspacing}
\begin{Verbatim}[commandchars=\\\{\}]
\PY{n}{q}\PY{o}{*}\PY{n}{p}
\end{Verbatim}
\end{tcolorbox}

            \begin{tcolorbox}[breakable, size=fbox, boxrule=.5pt, pad at break*=1mm, opacityfill=0]
\prompt{Out}{outcolor}{19}{\boxspacing}
\begin{Verbatim}[commandchars=\\\{\}]
 20i-26j-4k
\end{Verbatim}
\end{tcolorbox}
        
    \begin{tcolorbox}[breakable, size=fbox, boxrule=1pt, pad at break*=1mm,colback=cellbackground, colframe=cellborder]
\prompt{In}{incolor}{20}{\boxspacing}
\begin{Verbatim}[commandchars=\\\{\}]
\PY{p}{(}\PY{n}{q}\PY{o}{*}\PY{n}{p}\PY{p}{)}\PY{o}{.}\PY{n}{conjugate}\PY{p}{(}\PY{p}{)} \PY{o}{+} \PY{l+m+mi}{2}\PY{o}{*}\PY{p}{(}\PY{n}{q}\PY{o}{\PYZhy{}}\PY{l+m+mi}{3}\PY{o}{*}\PY{n}{p}\PY{p}{)}
\end{Verbatim}
\end{tcolorbox}

            \begin{tcolorbox}[breakable, size=fbox, boxrule=.5pt, pad at break*=1mm, opacityfill=0]
\prompt{Out}{outcolor}{20}{\boxspacing}
\begin{Verbatim}[commandchars=\\\{\}]
 -6-30i+12j-18k
\end{Verbatim}
\end{tcolorbox}
        
    \begin{tcolorbox}[breakable, size=fbox, boxrule=1pt, pad at break*=1mm,colback=cellbackground, colframe=cellborder]
\prompt{In}{incolor}{21}{\boxspacing}
\begin{Verbatim}[commandchars=\\\{\}]
\PY{n}{i} \PY{o}{=} \PY{n}{Quaternion}\PY{p}{(}\PY{l+m+mi}{0}\PY{p}{,}\PY{l+m+mi}{1}\PY{p}{,}\PY{l+m+mi}{0}\PY{p}{,}\PY{l+m+mi}{0}\PY{p}{)}
\PY{n}{j} \PY{o}{=} \PY{n}{Quaternion}\PY{p}{(}\PY{l+m+mi}{0}\PY{p}{,}\PY{l+m+mi}{0}\PY{p}{,}\PY{l+m+mi}{1}\PY{p}{,}\PY{l+m+mi}{0}\PY{p}{)}
\PY{n}{k} \PY{o}{=} \PY{n}{Quaternion}\PY{p}{(}\PY{l+m+mi}{0}\PY{p}{,}\PY{l+m+mi}{0}\PY{p}{,}\PY{l+m+mi}{0}\PY{p}{,}\PY{l+m+mi}{1}\PY{p}{)}
\end{Verbatim}
\end{tcolorbox}

    \begin{tcolorbox}[breakable, size=fbox, boxrule=1pt, pad at break*=1mm,colback=cellbackground, colframe=cellborder]
\prompt{In}{incolor}{22}{\boxspacing}
\begin{Verbatim}[commandchars=\\\{\}]
\PY{n}{i}\PY{o}{*}\PY{n}{i} \PY{o}{==} \PY{n}{j}\PY{o}{*}\PY{n}{j} \PY{o}{==} \PY{n}{k}\PY{o}{*}\PY{n}{k} \PY{o}{==} \PY{n}{i}\PY{o}{*}\PY{n}{j}\PY{o}{*}\PY{n}{k} \PY{o}{==} \PY{o}{\PYZhy{}}\PY{l+m+mi}{1}
\end{Verbatim}
\end{tcolorbox}

\begin{tcolorbox}[breakable, size=fbox, boxrule=.5pt, pad at break*=1mm, opacityfill=0]
\prompt{Out}{outcolor}{22}{\boxspacing}
\begin{Verbatim}[commandchars=\\\{\}]
True
\end{Verbatim}
\end{tcolorbox}
        
    La función que se encarga de crear el grupo de los Cuaternios es
\textit{QuaternionGroup}, donde únicamente se le ha de pasar por argumento
una de las dos representaciones siguientes:

    \begin{tcolorbox}[breakable, size=fbox, boxrule=1pt, pad at break*=1mm,colback=cellbackground, colframe=cellborder]
\prompt{In}{incolor}{23}{\boxspacing}
\begin{Verbatim}[commandchars=\\\{\}]
\PY{n}{Q} \PY{o}{=} \PY{n}{QuaternionGroup}\PY{p}{(}\PY{n}{rep}\PY{o}{=}\PY{l+s+s2}{\PYZdq{}}\PY{l+s+s2}{ijk}\PY{l+s+s2}{\PYZdq{}}\PY{p}{)}
\PY{n+nb}{print}\PY{p}{(}\PY{n}{Q}\PY{p}{)}
\end{Verbatim}
\end{tcolorbox}

\begin{tcolorbox}[breakable, size=fbox, boxrule=.5pt, pad at break*=1mm, opacityfill=0]
\prompt{Out}{outcolor}{23}{\boxspacing}
    \begin{Verbatim}[commandchars=\\\{\}]
Group with 8 elements: \{ 1,  i,  j,  k,  -k,  -j,  -i,  -1\}
    \end{Verbatim}
\end{tcolorbox}


    \begin{tcolorbox}[breakable, size=fbox, boxrule=1pt, pad at break*=1mm,colback=cellbackground, colframe=cellborder]
\prompt{In}{incolor}{24}{\boxspacing}
\begin{Verbatim}[commandchars=\\\{\}]
\PY{n}{Q2} \PY{o}{=} \PY{n}{QuaternionGroup}\PY{p}{(}\PY{n}{rep}\PY{o}{=}\PY{l+s+s2}{\PYZdq{}}\PY{l+s+s2}{permutations}\PY{l+s+s2}{\PYZdq{}}\PY{p}{)}
\PY{n+nb}{print}\PY{p}{(}\PY{n}{Q2}\PY{p}{)}
\end{Verbatim}
\end{tcolorbox}

    \begin{Verbatim}[commandchars=\\\{\}]
Group with 8 elements: \{(1, 4, 3, 2)(5, 7, 8, 6), (1, 7, 3, 6)(2, 8, 4, 5), (1, 6, 3, 7)(2, 5, 4, 8), (1, 8, 3, 5)(2, 6, 4, 7), (1, 2, 3, 4)(5, 6, 8, 7), (1, 5, 3, 8)(2, 7, 4, 6), (), (1, 3)(2, 4)(5, 8)(6, 7)\}
    \end{Verbatim}

    \begin{tcolorbox}[breakable, size=fbox, boxrule=1pt, pad at break*=1mm,colback=cellbackground, colframe=cellborder]
\prompt{In}{incolor}{25}{\boxspacing}
\begin{Verbatim}[commandchars=\\\{\}]
\PY{n}{Q}\PY{o}{.}\PY{n}{is\PYZus{}isomorphic}\PY{p}{(}\PY{n}{Q2}\PY{p}{)}
\end{Verbatim}
\end{tcolorbox}

            \begin{tcolorbox}[breakable, size=fbox, boxrule=.5pt, pad at break*=1mm, opacityfill=0]
\prompt{Out}{outcolor}{25}{\boxspacing}
\begin{Verbatim}[commandchars=\\\{\}]
True
\end{Verbatim}
\end{tcolorbox}
        

    
    
\item 
Por último, se ha programado la función \textit{QuaternionGroupGeneralised(n)} que define el grupo generalizado de los Cuaternios, con presentación:
\begin{align*}
    Q_n = \langle a,b \mid a^n = b^2, a^{2n}=1,
b^{-1}ab=a^{-1} \rangle \, .
\end{align*}
Cuando $n=2$ se tiene el grupo de los Cuaternios.
    \begin{tcolorbox}[breakable, size=fbox, boxrule=1pt, pad at break*=1mm,colback=cellbackground, colframe=cellborder]
\prompt{In}{incolor}{26}{\boxspacing}
\begin{Verbatim}[commandchars=\\\{\}]
\PY{n}{Q3} \PY{o}{=} \PY{n}{QuaternionGroupGeneralised}\PY{p}{(}\PY{n}{2}\PY{p}{)}
\PY{n}{Q3}\PY{o}{.}\PY{n}{is\PYZus{}isomorphic}\PY{p}{(}\PY{n}{Q}\PY{p}{)}
\end{Verbatim}
\end{tcolorbox}

\begin{tcolorbox}[breakable, size=fbox, boxrule=.5pt, pad at break*=1mm, opacityfill=0]
\prompt{Out}{outcolor}{26}{\boxspacing}
\begin{Verbatim}[commandchars=\\\{\}]
True
\end{Verbatim}
\end{tcolorbox}

\end{itemize}
        


    
    
    
    
  \newpage  
\item  \texttt{Dihedral.py}: del mismo modo que en el grupo de los
  Cuaternios, se ha realizado la implementación del grupo Diédrico en la
  clase Dihedral, \textit{class dihedral}. Ahora, un grupo Diédrico  $D_n$, de orden $2n$, almacenará internamente $n$ simetrías y  $n$ rotaciones que podrán representarse de tres formas equivalentes:

  \begin{enumerate}
  \item \textit{RS}: el conjunto de rotaciones serán denotadas por
    $R0, R1, \ldots, RN$ y las simetrías por $S1, S2,\ldots, SN$.
  \item \textit{Permutations}: se representará el grupo como un grupo de
    permutationes.
  \item \textit{Matrix}: se trata de una representación que hace
    referencia a la matriz del movimiento asociado.


    \begin{tcolorbox}[breakable, size=fbox, boxrule=1pt, pad at break*=1mm,colback=cellbackground, colframe=cellborder]
\prompt{In}{incolor}{27}{\boxspacing}
\begin{Verbatim}[commandchars=\\\{\}]
\PY{n}{D4} \PY{o}{=} \PY{n}{Dihedral}\PY{p}{(}\PY{l+m+mi}{4}\PY{p}{)}
\PY{n+nb}{print}\PY{p}{(}\PY{n}{D4}\PY{p}{)}
\end{Verbatim}
\end{tcolorbox}

    \begin{Verbatim}[commandchars=\\\{\}]
Rotaciones: \qquad  \qquad \qquad \quad Reflexiones:
(1.0, -0.0, 0.0, 1.0) \qquad \quad(1.0, 0.0, 0.0, -1.0)
(0.0, -1.0, 1.0, 0.0) \qquad \, (0.0, 1.0, 1.0, -0.0)
(-1.0, -0.0, 0.0, -1.0) \qquad (-1.0, 0.0, 0.0, 1.0)
(-0.0, 1.0, -1.0, -0.0) \qquad (-0.0, -1.0, -1.0, 0.0)
    \end{Verbatim}

    Para construir el grupo basta con llamar a la función $DihedralGroup$.
El primer parámetro $n$ hace referencia al grupo que se desea crear,
que tendrá orden $2n$, mientras que el segundo parámetro sirve para
indicar la representación deseada.

    \begin{tcolorbox}[breakable, size=fbox, boxrule=1pt, pad at break*=1mm,colback=cellbackground, colframe=cellborder]
\prompt{In}{incolor}{28}{\boxspacing}
\begin{Verbatim}[commandchars=\\\{\}]
\PY{n}{D} \PY{o}{=} \PY{n}{DihedralGroup}\PY{p}{(}\PY{l+m+mi}{3}\PY{p}{,} \PY{n}{rep}\PY{o}{=}\PY{l+s+s1}{\PYZsq{}}\PY{l+s+s1}{RS}\PY{l+s+s1}{\PYZsq{}}\PY{p}{)}
\PY{n+nb}{print}\PY{p}{(}\PY{n}{D}\PY{p}{)}
\end{Verbatim}
\end{tcolorbox}

    \begin{Verbatim}[commandchars=\\\{\}]
Group with 6 elements: \{'R1', 'R0', 'R2', 'S2', 'S1', 'S0'\}
    \end{Verbatim}

Comprobemos que efectivamente los grupos son isomorfos aunque su representación sea distinta:
    \begin{tcolorbox}[breakable, size=fbox, boxrule=1pt, pad at break*=1mm,colback=cellbackground, colframe=cellborder]
\prompt{In}{incolor}{29}{\boxspacing}
\begin{Verbatim}[commandchars=\\\{\}]
\PY{n}{D2} \PY{o}{=} \PY{n}{DihedralGroup}\PY{p}{(}\PY{l+m+mi}{3}\PY{p}{,} \PY{n}{rep}\PY{o}{=}\PY{l+s+s2}{\PYZdq{}}\PY{l+s+s2}{matrix}\PY{l+s+s2}{\PYZdq{}}\PY{p}{)}
\PY{n}{D3} \PY{o}{=} \PY{n}{DihedralGroup}\PY{p}{(}\PY{l+m+mi}{3}\PY{p}{,} \PY{n}{rep}\PY{o}{=}\PY{l+s+s2}{\PYZdq{}}\PY{l+s+s2}{permutations}\PY{l+s+s2}{\PYZdq{}}\PY{p}{)}
\end{Verbatim}
\end{tcolorbox}

    \begin{tcolorbox}[breakable, size=fbox, boxrule=1pt, pad at break*=1mm,colback=cellbackground, colframe=cellborder]
\prompt{In}{incolor}{30}{\boxspacing}
\begin{Verbatim}[commandchars=\\\{\}]
\PY{n}{D}\PY{o}{.}\PY{n}{is\PYZus{}isomorphic}\PY{p}{(}\PY{n}{D2}\PY{p}{)}\PY{p}{,} \PY{n}{D2}\PY{o}{.}\PY{n}{is\PYZus{}isomorphic}\PY{p}{(}\PY{n}{D3}\PY{p}{)}
\end{Verbatim}
\end{tcolorbox}


\begin{tcolorbox}[breakable, size=fbox, boxrule=.5pt, pad at break*=1mm, opacityfill=0]
\prompt{Out}{outcolor}{30}{\boxspacing}
\begin{Verbatim}[commandchars=\\\{\}]
(True, True)
\end{Verbatim}
\end{tcolorbox}
        
  \end{enumerate}

\end{itemize}
 

    % Add a bibliography block to the postdoc
    